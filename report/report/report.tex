\documentclass[twocolumn, a4paper]{article}

\usepackage{amsthm, amsmath, amssymb}
\usepackage{microtype}
\usepackage{graphicx}
\usepackage{booktabs} % for professional tables
\usepackage{hyperref}
\urlstyle{same}
\hypersetup{colorlinks=true, linkcolor=black, citecolor=black, urlcolor=blue}

\usepackage[top=2cm, bottom=2cm, left=1.5cm, right=1.5cm]{geometry}
\usepackage{titlesec}
    \titleformat{\title}{\large\bfseries}{}{}{}
    \titleformat{\section}{\normalfont\bfseries}{\thesection}{0.5em}{}
    \titleformat{\subsection}{\normalfont\it}{\thesubsection}{0.5em}{}
    \titleformat{\subsubsection}{\normalfont\normalsize\it}{\thesubsubsection}{0.5em}{}
    \titleformat{\paragraph}[runin]{\normalfont\bfseries}{\theparagraph}{0.5em}{}
    \titleformat{\subparagraph}[runin]{\normalfont\normalsize\it}{\thesubparagraph}{0.5em}{}
\usepackage[font=small,labelfont=bf,labelsep=space]{caption}

% \usepackage[ruled]{algorithm2e}
% \SetKwComment{Comment}{// }{ }

% \usepackage{tikz}
% \usetikzlibrary{shapes.geometric, arrows}
% \usetikzlibrary{calc}

% \tikzstyle{main_node} = [circle, minimum width=1cm,text centered, draw=black, fill=red!30]
% \tikzstyle{neigh_node} = [circle, minimum width=1cm,text centered, draw=black, fill=green!30]
% \tikzstyle{node} = [circle, minimum width=1cm,text centered, draw=black, fill=cyan!30]
% \tikzstyle{arrow} = [thick,->,>=stealth]


\newtheorem{theorem}{Theorem}
\newtheorem{lemma}{Lemma}
\newtheorem{corollary}{Corollary}
\theoremstyle{definition}
\newtheorem{definition}{Definition}

\begin{document}

\title{\bf\Large Segmentation of 3D volume images for connectomics}
\author{Oleh Shkalikov\texorpdfstring{ (5102818)
\\[0.7em]{\small Supervisors: Jannik Irmai, David Stein}
\\{\small Chairholder: Prof. Dr. Bjoern Andres}}{}}
\date{CMS Research project, TU Dresden}

\twocolumn[
    \begin{@twocolumnfalse}
        \maketitle

        \vspace{7ex}
    \end{@twocolumnfalse}
]

\section{Introduction}
The aim of this research project is to develop approaches
for segmentation of 3D volume images for connectomics given
a labeled dataset. The importance of this problem comes from the
developing methods (e.g. \cite{10.7554/eLife.25916}) of acquiring high resolution volume images
which makes hard to analyze all gathered data by hand because of their size.
Therefore automatic methods such as automatic labeling of data is needed.

In this work we will propose CNN based methods for solving this task and evaluate them
on the given dataset.

\section{Dataset description and analysis}

The raw data is available under
\url{https://www.epfl.ch/labs/cvlab/data/data-em/} and initially was used in
\cite{lucchi2011supervoxel,lucchi2013learning}.

\section{Related work}

\section{Methodology}

\subsection{Probation of classical approaches}

\subsection{Convolutional neural networks}

\subsection{Pretraining}

\subsection{Postprocessing}

\section{Experiments}

\subsection{Efficiency of pretraining}

\subsection{Segmentation models evaluation}

\section{Directions of further research}
\subsection{Context-free formal grammars}

\section{Conclusions}

\bibliography{../references.bib}
\bibliographystyle{ieeetr}

\onecolumn
\section*{Appendix}

\end{document}