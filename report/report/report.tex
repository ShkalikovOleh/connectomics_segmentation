\documentclass[twocolumn, a4paper]{article}

\usepackage{amsthm, amsmath, amssymb}
\usepackage{microtype}
\usepackage{graphicx}
\usepackage{booktabs} % for professional tables
\usepackage{hyperref}
\usepackage{multirow}
\urlstyle{same}
\hypersetup{colorlinks=true, linkcolor=black, citecolor=black, urlcolor=blue}

\usepackage[top=2cm, bottom=2cm, left=1.5cm, right=1.5cm]{geometry}
\usepackage{titlesec}
    \titleformat{\title}{\large\bfseries}{}{}{}
    \titleformat{\section}{\normalfont\bfseries}{\thesection}{0.5em}{}
    \titleformat{\subsection}{\normalfont\it}{\thesubsection}{0.5em}{}
    \titleformat{\subsubsection}{\normalfont\normalsize\it}{\thesubsubsection}{0.5em}{}
    \titleformat{\paragraph}[runin]{\normalfont\bfseries}{\theparagraph}{0.5em}{}
    \titleformat{\subparagraph}[runin]{\normalfont\normalsize\it}{\thesubparagraph}{0.5em}{}
\usepackage[font=small,labelfont=bf,labelsep=space]{caption}

% \usepackage[ruled]{algorithm2e}
% \SetKwComment{Comment}{// }{ }

% \usepackage{tikz}
% \usetikzlibrary{shapes.geometric, arrows}
% \usetikzlibrary{calc}

% \tikzstyle{main_node} = [circle, minimum width=1cm,text centered, draw=black, fill=red!30]
% \tikzstyle{neigh_node} = [circle, minimum width=1cm,text centered, draw=black, fill=green!30]
% \tikzstyle{node} = [circle, minimum width=1cm,text centered, draw=black, fill=cyan!30]
% \tikzstyle{arrow} = [thick,->,>=stealth]


\newtheorem{theorem}{Theorem}
\newtheorem{lemma}{Lemma}
\newtheorem{corollary}{Corollary}
\theoremstyle{definition}
\newtheorem{definition}{Definition}

\begin{document}

\title{\bf\Large Segmentation of 3D volume images for connectomics}
\author{Oleh Shkalikov\texorpdfstring{ (5102818)
\\[0.7em]{\small Supervisors: Jannik Irmai, David Stein}
\\{\small Chairholder: Prof. Dr. Bjoern Andres}}{}}
\date{CMS Research project, TU Dresden}

\twocolumn[
    \begin{@twocolumnfalse}
        \maketitle

        \vspace{7ex}
    \end{@twocolumnfalse}
]

\section{Introduction}
The aim of this research project is to develop approaches
for segmentation of 3D volume images for connectomics given
a labeled dataset. The importance of this problem comes from the
developing of methods (e.g. \cite{10.7554/eLife.25916}) of acquiring high resolution volume images
which makes hard to analyze all gathered data by hand because of their size.
Therefore automatic methods such as automatic labeling of data is needed.

In this work we will propose CNN based methods for solving this task and evaluate them
on the given dataset.

\section{Dataset description and analysis}

The labeled dataset which we use for out project is based on
the image volumes of the CA1 hippocampus region of the brain
acquired by a focused ion beam scanning electron microscope (FIB-SEM).
The raw data is available under
\url{https://www.epfl.ch/labs/cvlab/data/data-em/} and initially was used in
\cite{lucchi2011supervoxel,lucchi2013learning}. This dataset also contains binary
labels for mitochondria which we don't use because we have been provided with wider
set of labels.

The actual labeling has been performed by An Dang Thanh (student of the MLCV lab)
only on slices of subvolumes where to
each pixel of slice \textbf{one} of the following labels has been assigned:
\begin{itemize}
    \item cell membrane
    \item cell cytoplasm
    \item mitochondrion
    \item mitochondrion membrane
    \item synapse
    \item vesicle
    \item undefined (in the case where annotator was uncertain about the correct label)
\end{itemize}
In total 9 slices of the raw data volume has been labeled and split into train,
validation and test splits: 3 slices for each split, but every train slice has a shape
\( 600 \times 600 \) whereas every validation and test slice has a shape \( 425 \times 425 \).
The locations of slices in the raw volume are denoted in the table~\ref{tab:data_loc}.
\begin{table}[h]
    \centering
    \begin{tabular}{c|c|c|c}
        Split                             & X range           & Y range            & Z range           \\
        \hline

        \multirow[vpos]{3}{*}{Train}      & \( [510, 1109] \) & \( [200, 799] \)   & 200               \\
                                          & \( [510, 1109] \) & 510                & \( [150, 749 ] \) \\
                                          & \( 595  \)        & \( [200, 799] \)   & \( [150, 749 ] \) \\
        \hline
        \multirow[vpos]{3}{*}{Validation} & \( [675, 1099] \) & \( [1000, 1424] \) & 320               \\
                                          & \( [675, 1099] \) & 1140               & \( [200, 624] \)  \\
                                          & \( 985  \)        & \( [1000, 1424] \) & \( [200, 624] \)  \\
        \hline
        \multirow[vpos]{3}{*}{Test}       & \( [510, 1109] \) & \( [200, 799] \)   & 200               \\
                                          & \( [510, 1109] \) & 510                & \( [150, 749 ] \) \\
                                          & \( 595  \)        & \( [200, 799] \)   & \( [150, 749 ] \) \\
        \hline
    \end{tabular}
    \caption{Location of the labeled slices in the raw volume}
    \label{tab:data_loc}
\end{table}

\section{Related work}

\section{Methodology}

\subsection{Probation of classical approaches}

\subsection{Convolutional neural networks}

\subsection{Pretraining}

\subsection{Postprocessing}

\section{Experiments}

\subsection{Efficiency of pretraining}

\subsection{Segmentation models evaluation}

\section{Directions of further research}
\subsection{Context-free formal grammars}

\section{Conclusions}

\bibliography{../references.bib}
\bibliographystyle{ieeetr}

\onecolumn
\section*{Appendix}

\end{document}